\section{Wigner distribution}

The Wigner distribution of a spin-less particule is defined as
\begin{equation}
	\label{eq:Wigner}
	w(\vect{r}, \vect{p}) = \frac{1}{(2\pi\hbar)^3} \int \ud\vect{u} \, \rho\left( \vect{r}-\frac{\vect{u}}{2} , \vect{r}+\frac{\vect{u}}{2} \right) \expcompl{\vect{u}.\vect{p}/\hbar} ,
\end{equation}
where $\rho(\vect{u},\vect{v}) = \bra{\vect{u}} \hat{\rho} \ket{\vect{v}}$ is the density matrix in real space.

In the following, we are going to explicitly compute this Wigner distribution in the real and reciprocal space representations.
We will focus on pure states only since this is enough to compute the Wigner distribution of any statistical mixture. Indeed, if the system is in a statistical mixture of pure states, the density matrix is just the linear superposition of pure states density matrices $\hat{\rho} = \sum_i \pi_i \hat{\rho}_i$ where $\pi_i$ are probabilities. The Wigner distribution is then also the linear superposition of the pure states Wigner distributions.
We will also derive its motion equation.



\subsection{Real space representation}

We now assume that the system is in a pure state $\ket{\psi}$.
From its real space representation in Eq.~(\ref{eq:decomposition}) we compute the real space representation of the density matrix
\begin{equation}
	\label{eq:rDensityMatrix}
	\hat{\rho} = \int \ud\vect{r'}\ud\vect{r''} \psi(\vect{r'}) \psi(\vect{r''})^{\ast} \ket{\vect{r'}}\bra{\vect{r''}} .
\end{equation}
Injecting Eq.~(\ref{eq:rDensityMatrix}) into Eq.~(\ref{eq:Wigner}) yields
\begin{equation}
	\label{eq:rWigner}
	w(\vect{r}, \vect{p}) = \frac{1}{(2\pi\hbar)^3} \int \ud\vect{u} \, \psi\left( \vect{r}-\frac{\vect{u}}{2} \right) \psi\left( \vect{r}+\frac{\vect{u}}{2} \right) \expcompl{\vect{u}.\vect{p}/\hbar} .
\end{equation}



\subsection{Reciprocal space representation}

We now decompose the pure state $\ket{\psi}$ on the momentum basis in the continuum limit.
Using Eq.~(\ref{eq:decomposition}), the density matrix reads
\begin{equation}
	\label{eq:pDensityMatrix}
	\hat{\rho} = \int \ud\vect{p'}\ud\vect{p''} \psi(\vect{p'}) \psi(\vect{p''})^{\ast} \ket{\vect{p'}}\bra{\vect{p''}} .
\end{equation}
Injecting Eq.~(\ref{eq:pDensityMatrix}) into Eq.~(\ref{eq:Wigner}) and using relation~(\ref{eq:pDirac}) yields
\begin{equation}
	\label{eq:pWigner}
	w(\vect{r}, \vect{p}) = \frac{1}{(4\pi\hbar)^3} \int \ud\vect{q} \, \tilde{\psi}\left( \vect{p}+\frac{\vect{q}}{2} \right) \tilde{\psi}\left( \vect{p}-\frac{\vect{q}}{2} \right) \expcompl{\vect{r}.\vect{q}/\hbar} ,
\end{equation}
where we had to exchange real and reciprocal space integrations.



\subsection{Motion equation}

We now focus on the evolution of the Wigner distribution. We first derive a general differential (in the absence of an electromagnetic field) and compare it with its classical counterpart, the Liouville equation. We then consider simple examples where the two equations are identical.


\subsubsection{Equation}

Let's consider a particule with mass $m$ subject to potential $V(\vect{r})$ initially in the state $\ket{\psi}$. In the Schrodinger representation the density matrix is the solution of
\begin{equation}
	\label{eq:evolutionDensityMatrix}
	\diff{}{t} \hat{\rho} = \frac{1}{\ic\hbar}\commute{\hat{H}}{\hat{\rho}} ,
\end{equation}
with initial condition $\rho(t=0) = \ket{\psi}\bra{\psi}$ and with Hamiltonian
\begin{equation}
	\label{eq:Hamiltonian}
	\hat{H} = \frac{\hat{\vect{p}}^2}{2m} + V(\hat{\vect{r}}) .
\end{equation}

We first compute the time evolution due to the kinetic part of the Hamiltonian, \textit{i.e.}, we compute the evolution of a free particle.
The commutator in Eq.~(\ref{eq:evolutionDensityMatrix}) is
\begin{equation}
	\commute{\hat{H}_{\text{free}}}{\hat{\rho}} =
	\frac{1}{2m} \int \ud\vect{p'}\ud\vect{p''} \, \tilde{\psi}(\vect{p'}) \tilde{\psi}(\vect{p''})^{\ast} \ket{\vect{p'}} \bra{\vect{p''}} \left(\vect{p'}^2 - \vect{p''}^2\right) ,
	\nonumber
\end{equation}
where time dependence of $\tilde{\psi}$ is implicit.
The time derivative of the Wigner distribution is then
\begin{equation}
	\left.\dpart{w}{t}\right|_{\text{free}} = 
	- \frac{\vect{p}}{m} \left[ \frac{\hbar}{\ic} \frac{1}{(4\pi\hbar)^3}
	\int \ud\vect{q} \,\vect{q} \, \tilde{\psi}\left( \vect{p}+\frac{\vect{q}}{2} \right) \tilde{\psi}\left( \vect{p}-\frac{\vect{q}}{2} \right) \expcompl{\vect{r}.\vect{q}/\hbar} \right] .
	\nonumber
\end{equation}
From Eq.~(\ref{eq:pWigner}) we can see that the integral in the right-hand side is proportional to the gradient (over real space coordinates) of the Wigner distribution.
The Wigner distribution of a single free particle thus satisfies
\begin{equation}
	\label{eq:freeEvolution}
	\left.\dpart{w}{t}\right|_{\text{free}} + \frac{1}{m} \vect{p}.\nabla_{\vect{r}} w = 0 .
\end{equation}
We now compute the time evolution due to the potential.
The commutator between the potential and the density matrix is
\begin{equation}
	\commute{V(\hat{\vect{r}})}{\hat{\rho}} =
	\int \ud\vect{r'}\ud\vect{r''} \, \psi(\vect{r'}) \psi(\vect{r''})^{\ast} \ket{\vect{r'}} \bra{\vect{r''}} \left(V(\vect{r'}) - V(\vect{r''})\right) .
	\nonumber
\end{equation}
The potential's contribution to the Wigner distribution time evolution is then
\begin{equation}
	\left.\dpart{w}{t}\right|_{\text{potential}} =
	\frac{1}{\ic\hbar} \frac{1}{(2\pi\hbar)^3}
	\int \ud\vect{u} \, \psi\left( \vect{r}-\frac{\vect{u}}{2} \right) \psi\left( \vect{r}+\frac{\vect{u}}{2} \right)^{\ast} \, \left[ V\left( \vect{r}-\frac{\vect{u}}{2} \right) - V\left(\vect{r}+\frac{\vect{u}}{2}\right) \right] \expcompl{\vect{u}.\vect{p}/\hbar}
	\nonumber
\end{equation}
where we recognize the Fourier transform of a product. Using Eq.~(\ref{eq:convolutionFourier}) we can express the right-hand side of the previous expression as a convolution product of the Wigner distribution and a Kernel
\begin{equation}
	\label{eq:potentialEvolution}
	\left.\dpart{w}{t}\right|_{\text{potential}} =
	\int \ud\vect{q} \, \mathcal{N}(\vect{r},\vect{q}) w(\vect{r}, \vect{p}-\vect{q}) ,
\end{equation}
where the Kernel $\mathcal{N}$ is given by
\begin{equation}
	\label{eq:kernel}
	\mathcal{N}(\vect{r},\vect{q}) =
	\frac{1}{\ic\hbar} \frac{1}{(2\pi\hbar)^3} \int \ud\vect{u} \, \left[ V\left(\vect{r}-\frac{\vect{u}}{2}\right) - V\left( \vect{r} + \frac{\vect{u}}{2} \right) \right] \expcompl{\vect{u}.\vect{q}/\hbar} .
\end{equation}
Finally, the time evolution of the Wigner distribution of a particle with Hamiltonian (\ref{eq:Hamiltonian}) is
\begin{equation}
	\label{eq:evolution}
	\dpart{w}{t} + \frac{1}{m} \vect{p}.\nabla_{\vect{r}} w =
	\int \ud\vect{q} \, \mathcal{N}(\vect{r},\vect{q}) w(\vect{r}, \vect{p}-\vect{q}) .
\end{equation}
This result can be compared to its classical counterpart: the probability distribution $f$ of a single particle with (classical) Hamiltonian (\ref{eq:Hamiltonian}) obeys the Liouville equation
\begin{equation}
	\label{eq:Liouville}
	\dpart{f}{t} + \nabla_{\vect{p}} H . \nabla_{\vect{r}} f - \nabla_{\vect{r}} H . \nabla_{\vect{p}} f = 0 .
\end{equation}
The first two terms of equations (\ref{eq:evolution}) and (\ref{eq:Liouville}) have the same structure, \textit{i.e.} the two distributions obey the same equation for a free particle. Only the third term, \textit{i.e.} the potential contribution, is different.


\subsubsection{`Classical' potentials}

We now consider three particular potentials where equations (\ref{eq:evolution}) and (\ref{eq:Liouville}) have exactly the same structure: linear, quadratic and slowly varying potentials.

A general linear potential is defined as
\begin{equation}
	V(\vect{r}) = \vect{a}.(\vect{r}-\vect{r}_0) .
	\nonumber
\end{equation}
Its kernel is
\begin{equation}
	\label{eq:linearKernel}
	\mathcal{N}(\vect{r},\vect{q}) =
	\vect{a}.\nabla_{\vect{q}} \delta(\vect{q}) .
\end{equation}
With $\vect{a} = \nabla_{\vect{r}}V$ and integrating by part, motion equation for the Wigner distribution reduces to the Liouville equation.

The case of the quadratic potential is very similar.
Its general form is
\begin{equation}
	V(\vect{r}) = (\vect{r}-\vect{r}_0) A (\vect{r}-\vect{r}_0) ,
	\nonumber
\end{equation}
where $A$ is a positive symmetric matrix. The positivity ensures that the potential is attractive in all directions.
Its kernel is
\begin{equation}
	\label{eq:quadraticKernel}
	\mathcal{N}(\vect{r},\vect{q}) =
	(\vect{r}-\vect{r}_0) A \nabla_{\vect{q}} \delta(\vect{q}) .
\end{equation}
With $A (\vect{r}-\vect{r}_0) = \nabla_{\vect{r}}V$ and the same integration by part as for the linear potential, motion equation for the Wigner distribution reduces to the Liouville equation.

